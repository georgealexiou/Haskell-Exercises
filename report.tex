\documentclass{article}
\usepackage[utf8]{inputenc}

\title{COMP2209 - Lambda Calculus Coursework}
\author{Georgios Alexiou - ga4g18}
\date{January 2020}

\begin{document}

\maketitle

In order to complete the COMP2209 Lambda Calculus Coursework, 
a variety of development techniques and tools where used. When taking on a
challenge, I examined the provided sample test cases in \textit{Tests.hs} in
order to gain a good understanding of what my code needs to accomplish. After writing
code that covers the base test cases, I used test cases from other challenges as well as my own,
in order to to further examine the effectiveness of my code. This way, I was able to
develop code that worked for a variety of lambda as well as let expressions for each
given problem.  
\\\\
\indent The recursion coding technique was heavily used through most of the examples,
as each lambda or let expression given was thought of as a syntax tree which was
recursively traversed to give the desired result. For a lot of examples, a combination
of recursion and pattern matching, helped seperate different types of lambda or let subexpressions.
In order to create simpler and cleaner code, all challenges consist of smaller functions
that aim to complete simpler tasks to lead to the desired result. I also heavily commented
my code so as to provide anyone reading it as well as myself with a clear idea of what
each function does but also how it operates.
\\\\
\indent When writing my code, the standard Haskell debugging tools where heavily employed
in order to locate errors and reconsider the techniques used. This was especially important for
Challenges 1 and 2 where a lot of the time it was hard to locate the errors that occured when alpha
or beta reducing in each respective example.z

\end{document}
